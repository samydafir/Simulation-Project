\documentclass[12pt,a4paper]{article}
\usepackage[utf8]{inputenc}
\usepackage{amsmath}
\usepackage{amsfonts}
\usepackage{amssymb}
\usepackage{graphicx}
\usepackage[ngerman]{babel}
\usepackage{inputenc}

\title{Farbdrucker Scheduling}
\author{Dominik Baumgartner, Thomas Samy Dafir, Laurentiu Vlad}
\date{}

\begin{document}
	\maketitle
	
	\section{Aufgabenstellung}
	Es soll ein Szenario simuliert werden, bei dem 2 Farbdrucker mehreren Usern zur Verfügung stehen und in gewissen Zeitabständen Print-Jobs mit
	verschiedenen Prioritäten und Abarbeitungszeiten erstellt werden. Insgesamt gibt es 3 Prioritätslevels: Student (niedrig), Professor (mittel), System (hoch). Jobs 
	werden in Queues gesammelt und von beiden Druckern abgearbeitet. Dabei ist es immer möglich, dass jobs mit niedrigerer Priorität während ihrer Bearbeitung in 
	einem Drucker von Jobs höherer Priuorität unterbrochen werden können, sprich Professoren-Jobs unterbrechen Studenten-Jobs, System-jobs unterbrechen alle 
	anderen. Unterbrochene Jobs werden so bald als möglich wieder fortgesetzt.
	
	\section{Modell}
	Um den Sachverhalt aus der Aufgabenstellung simulieren zu können wurde ein Modell bestehend aus 6 Prozessen und einer Model-Klasse erstellt (+ 1 weiterer Prozess für die Erweiterung).
	Im folgenden wird kurz auf die Aufgabe eines jeden Prozesses eingegangen und am Schluss das Zusammenspiel der einzelnen Prozesse erklärt. Außerden gehen wir noch kurz auf die Veränderungen
	ein, die nötig waren, um unsere Erweiterung zu realisieren.
	
	\subsection{NewJobProcess}
	Wie in der Aufgabenstellung beschrieben, erhalten unsere beiden Drucker in gewissen Zeitabständen Druckaufträge unterschiedlicher Priorität. Diese werden vermutlich von Usern oder den System
	in auftrag gegeben. Um dies zu simulieren brauchen wir natürlich eine Art ``Job-Generator''. Diese Aufgabe wird vom $NewJobProcess$ übernommen. NewJobProcess läuft während der gesamten 
	Simulationsdauer und erstellt in gewissen Zeitabständen neue Jobs mit einer bestimmten Ausführungsdauer. Danach wird für eine gewisse Zeit gewartet und danach wieder ein Job erstellt. Die Warte-
	und Ausführungszeiten werden von einem $ContDistUniform$-Objekt im Modell bezogen. Ein erstellter Prozess wird sofort aktiviert. Da wir 3 verschiene Arten von Jobs simulieren, benötigen wir auch
	3 Instanzen dieser Klasse.
	
	\subsection{JobProcess}
		
	
	\section{Ergebnisse}
	
	








\end{document}